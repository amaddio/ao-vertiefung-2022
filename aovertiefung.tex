%%
%% This is file `./samples/longsample.tex',
%% generated with the docstrip utility.
%%
%% The original source files were:
%%
%% apa7.dtx  (with options: `longsample')
%% ----------------------------------------------------------------------
%% 
%% apa7 - A LaTeX class for formatting documents in compliance with the
%% American Psychological Association's Publication Manual, 7th edition
%% 
%% Copyright (C) 2019 by Daniel A. Weiss <daniel.weiss.led at gmail.com>
%% 
%% This work may be distributed and/or modified under the
%% conditions of the LaTeX Project Public License (LPPL), either
%% version 1.3c of this license or (at your option) any later
%% version.  The latest version of this license is in the file:
%% 
%% http://www.latex-project.org/lppl.txt
%% 
%% Users may freely modify these files without permission, as long as the
%% copyright line and this statement are maintained intact.
%% 
%% This work is not endorsed by, affiliated with, or probably even known
%% by, the American Psychological Association.
%% 
%% ----------------------------------------------------------------------
%% 
\documentclass[man]{apa7}

\usepackage{lipsum}

\usepackage[american]{babel}

\usepackage{csquotes}
\usepackage[style=apa,sortcites=true,sorting=nyt,backend=biber]{biblatex}
\DeclareLanguageMapping{american}{american-apa}
\addbibresource{bibliography.bib}

% ellipses (three dots) in quatations definitions from: https://walden-family.com/public/texland/ellipses.pdf
    %dots for main text
    \def\bigdotsspace{3pt}
    %three dots
    \def\mydots{\hbox{\hspace{\bigdotsspace}.\hspace{\bigdotsspace}.\hspace{\bigdotsspace}.\hspace{\bigdotsspace}\hspace{\bigdotsspace}}}
    %I like the same size space on each side of the ellipsis as
    % is between the dots of the ellipsis
    
    \hypersetup{
        nolinks=true,
        colorlinks=false,
        linkcolor=blue,
        filecolor=magenta,      
        urlcolor=cyan,
        pdftitle={Development of Media Richness Perception},
        pdfpagemode=FullScreen,
        % to turn off warning: "Package hyperref Warning: Draft mode on." add empty final attribute
        % final
    }

\title{Development of Media Richness Perception Over Time and it's Effect on Well-Being and Communication Efficacy}
% title option 2: \title{Changes in Perceived Media Richness over time and it's effect on well-being and communication efficacy}
\shorttitle{Media Richness Perception}

\author{Antonio Amaddio}
\affiliation{Freie Universität Berlin \\ A\&O Vertiefung, Winter 2022/23, Supervisor: Dr. Lisa Handke}

\leftheader{Amaddio}

% \abstract{\mbox{}}

\keywords{Media Richness, Compensatory Adaption, Media Naturalness, Channel Expansion, Well-Being, Communication Effectiveness}

%\authornote{
%   \addORCIDlink{Antonio Amaddio}{0000-0000-0000-0000}

  %Correspondence concerning this article should be addressed to Daniel A. Weiss, Department of Educational Psychology, Counseling and
  %Special Education, A University Somewhere, 123 Main St., Oneonta, NY
  %13820.  E-mail: daniel.weiss.led@gmail.com}

\begin{document}
\maketitle
Pandemic related changes in work-life have sparkled a flood of research investigating the psychological effects of remote work. Videoconferencing turns out to be one of the most used communication media to bridge a risky face-to-face meeting to a non-risky video based alternative \parencite{Riedl2021}.  Research shares the belief that videoconferencing has a fatiguing psychological effect on its users. The so called \textit{Zoom-Fatigue} seems to be a common reaction to the new media use. The daunting effect is attributed to cue characteristics which are not natural compared to a face-to-face conversation (see \cite{Riedl2021}). Our perception has been evolutionary evolved by natural social communication and its cues (verbal, non-verbal, mimic, gesture etc.). If humans are capable to evolve in that sense, why wouldn't humans also adjust to new communication characteristics over time? Therefore, I investigate the experience of one individual in a single case study to explore potential changes in perception and behavior towards videoconferencing over time.

\section{Theoretical Background}

In 1983, \citeauthor{daft1983information} proposed a theory that describes a media's ability to support communication on a richness scale. That is, how helpful the use of e.g. a telephone is to transport information required to "reduce uncertainty and clarify ambiguity" (p 5). Media is rated from rich to lean, where Face-to-Face is the richest and numerical formal computer output is lowest in richness. The theory objectively assigns richness to a media, even though they mention themselves that richness is somewhat dependent on the task requirement to "provide new substantial understanding" (p. 7). Looking at organizations that work fully/partially remote, this categorization does not make sense anymore at first sight. Given that teams in these organizations (1) are still functioning and (2) provide sufficient labor for the organization to economically survive.

According to \citeauthor{Kock2005}, this lack of diminished outcome quality is explainable by the workers' efforts to compensate for hindering characteristics of a medium. The two theories are not contradictory. Instead, \citeauthor{Kock2005} supports by suggesting that the media barriers increase communication ambiguity and required cognitive effort to process relevant pieces of information. He calls workers' behavior to "overcompensate" with lean media information processing: Compensatory Adaption (p. 270). If the assumptions are true, then the (1) increased cognitive demand should take effect on a worker's well-being and (2) the compensatory adaption should have a null effect on team efficacy, i.e. not diminishing.

Most of the studies undertaken regarding the fatiguing effect of videoconferencing were one-time data collections. This case study explores changes of this effect over time. This means that the variable of interest must allow changes to the media richness. \cite{Carlson1999} considers media richness as a subjective variable–controversy to \citeauthor{daft1983information}. The authors argue that richness is a personal (subjective) evaluation that one derives from experience. More specifically, experience with the (1) communication partner, (2) channel (media), (3) its topic and the (4) organizational context form a knowledge base. This knowledge can be used to adequately decode information even though the communication media does not naturally support to transfer relevant cues. Similarly, data can be enriched from the sender by encoding information that alleviates understanding by the receiver–which could not be inferred rationally from the raw media message. To sum up: Experience leads to knowledge and shape skills, which forms perception of media richness.

A potential scenario of changes in perception is given: A person starts using videoconferencing software. The person struggles to maintain speaking order and the voices overlap annoyingly. The person consideres the medium as rather lean as it lacks natural cues which are present in a face-to-face conversation. Over time, the same person raises a virtual hand to enhance speaking-order and diminish confusion (information encoding). In the meantime the person perceives videoconferencing as richer than before. As a listener, the same person was distracted by the mirroring camera and could not focus on the other people. Over time, the person turns of the mirroring camera to reduce self-attention biases (decode) (see also \cite{Riedl2021}).

--
If decoding example does not make sense: use smiley as emotion expression. The person finds it daunting to scan each each individual for facial expression. The person has learned that emotions are frequently expressed with emojis.
--

Taken together, it is hypothesized that, a developed knowledge base of functional behaviors increases the perceived richness of a media and reduces a remote worker's cognitive effort, hence well-being.

Furthermore, a developed knowledge base of functional behaviors should maintain (or increase) communication efficacy even though communication is based remotely on videoconferencing software.

Note: In this case study, only learning experiences in encoding and decoding infomration are respected that effect perceived Media Richness.

\section{Case Study}

The interviewed person is of female sex, 34 years old, married and has one son (17 months). She graduated from a hotel management school, worked in hotels ever since as a senior convention sales manager. Because of the pandemic crisis and flat hotel booking rates, she joined a more secure job in a tech start-up 2.5 years ago where she was 12 months on parental leave. The company uses \textit{Gather.town} as primary video-conferencing tool. She spends 25\% of her time communicating in Gather.

The interview took place on 01-30-2023 at 10:30am and lasted $\approx$55 minutes. The interviewee reported preferring face-to-face conversations and to not spend more time than necessary in front of a computer or smartphone. Digital communication in her previous work experience was limited to email and telephone. She has been chosen because she represents an en/decoding expert in natural face-to-face conversations, but a person with a novice-level of videoconferencing skills in a remote work scenario. 

\subsection{Ergebnisse/Analyse (2 - 3 Seiten)}
• Darstellung der für die Beantwortung der Fragestellung relevanten Ergebnisse
• Ergebnisse sollten nicht nur aufgezählt, sondern in irgendeiner Form
konsolidiert/kategorisiert/abstrahiert werden, durch entsprechende Strukturierung im Text und/oder in Form von Tabellen oder Abbildungen
o Dies können auch Abwandlungen bereits etablierter Modelle/Abbildungen sein
• Einordnung der Ergebnisse in gängige Modelle/Theorien bzw. Erklärung der Befunde durch
bestehende Forschung
z.B. wie manifestiert sich hohe Autonomie bei der Person, was löst dies für Herausforderungen aus, die sich auf diese erhöhte Autonomie auswirken (z.B. die Erwartungshaltung, durch die hohe Flexibilität könnten „zwischendurch“ auch leicht Einkäufe erledigt werden; ständige Unterbrechungen durch Familienmitglieder, die dazu führen, dass man in seinem Arbeitsfluss immer wieder unterbrochen wird bzw. Probleme hat, sich zu konzentrieren); wie passen diese Schilderungen zu bekannten Konstrukten aus der Forschung (z.B. Family-to-Work Conflict); welche Auswirkungen haben diese Herausforderungen/wie wirken Sie sich laut Wahrnehmung der interviewten Person selbst bzw. aus Ihrer Interpretation deren Schilderungen auf deren Wohlbefinden (und welche Facetten davon genau?) aus? Lassen sich noch weitere z.B. moderierende Faktoren aus den Schilderungen der interviewten Person identifizieren (z.B., dass die hohe Flexibilität eigentlich nur dann problematisch ist, wenn die Familienmitglieder zu Hause sind)
ACHTUNG: Ihr werdet sicherlich nicht innerhalb eures individuellen Fallberichts so viele Punkte (vgl. Abbildung) adressieren, d.h. dies sind nur Beispiele, es wäre auch völlig in Ordnung (bzw. sogar eher erwünscht), wenn ihr euch nur auf 2-3 unterschiedliche Konstrukte konzentriert und diese aber dafür mehr ausbaut.

\subsection{Fazit/Konklusion (1⁄2 Seite)}
• Beantwortung der Fragestellung
• Kurze, zusammenfassende Einordnung der Ergebnisse ins übergeordnete Thema

\printbibliography

\appendix

\end{document}

%% 
%% Copyright (C) 2019 by Daniel A. Weiss <daniel.weiss.led at gmail.com>
%% 
%% This work may be distributed and/or modified under the
%% conditions of the LaTeX Project Public License (LPPL), either
%% version 1.3c of this license or (at your option) any later
%% version.  The latest version of this license is in the file:
%% 
%% http://www.latex-project.org/lppl.txt
%% 
%% Users may freely modify these files without permission, as long as the
%% copyright line and this statement are maintained intact.
%% 
%% This work is not endorsed by, affiliated with, or probably even known
%% by, the American Psychological Association.
%% 
%% This work is "maintained" (as per LPPL maintenance status) by
%% Daniel A. Weiss.
%% 
%% This work consists of the file  apa7.dtx
%% and the derived files           apa7.ins,
%%                                 apa7.cls,
%%                                 apa7.pdf,
%%                                 README,
%%                                 APA7american.txt,
%%                                 APA7british.txt,
%%                                 APA7dutch.txt,
%%                                 APA7english.txt,
%%                                 APA7german.txt,
%%                                 APA7ngerman.txt,
%%                                 APA7greek.txt,
%%                                 APA7czech.txt,
%%                                 APA7turkish.txt,
%%                                 APA7endfloat.cfg,
%%                                 Figure1.pdf,
%%                                 shortsample.tex,
%%                                 longsample.tex, and
%%                                 bibliography.bib.
%% 
%%
%% End of file `./samples/longsample.tex'.
