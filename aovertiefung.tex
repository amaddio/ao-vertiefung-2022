%%
%% This is file `./samples/longsample.tex',
%% generated with the docstrip utility.
%%
%% The original source files were:
%%
%% apa7.dtx  (with options: `longsample')
%% ----------------------------------------------------------------------
%% 
%% apa7 - A LaTeX class for formatting documents in compliance with the
%% American Psychological Association's Publication Manual, 7th edition
%% 
%% Copyright (C) 2019 by Daniel A. Weiss <daniel.weiss.led at gmail.com>
%% 
%% This work may be distributed and/or modified under the
%% conditions of the LaTeX Project Public License (LPPL), either
%% version 1.3c of this license or (at your option) any later
%% version.  The latest version of this license is in the file:
%% 
%% http://www.latex-project.org/lppl.txt
%% 
%% Users may freely modify these files without permission, as long as the
%% copyright line and this statement are maintained intact.
%% 
%% This work is not endorsed by, affiliated with, or probably even known
%% by, the American Psychological Association.
%% 
%% ----------------------------------------------------------------------
%% 
\documentclass[man]{apa7}

\usepackage{lipsum}

\usepackage[american]{babel}

\usepackage{csquotes}
\usepackage[style=apa,sortcites=true,sorting=nyt,backend=biber]{biblatex}
\DeclareLanguageMapping{american}{american-apa}
\addbibresource{bibliography.bib}

% ellipses (three dots) in quatations definitions from: https://walden-family.com/public/texland/ellipses.pdf
    %dots for main text
    \def\bigdotsspace{3pt}
    %three dots
    \def\mydots{\hbox{\hspace{\bigdotsspace}.\hspace{\bigdotsspace}.\hspace{\bigdotsspace}.\hspace{\bigdotsspace}\hspace{\bigdotsspace}}}
    %I like the same size space on each side of the ellipsis as
    % is between the dots of the ellipsis
    
    \hypersetup{
        nolinks=true,
        colorlinks=false,
        linkcolor=blue,
        filecolor=magenta,      
        urlcolor=cyan,
        pdftitle={Effect of Personality Composition in Student Teams},
        pdfpagemode=FullScreen,
        % to turn off warning: "Package hyperref Warning: Draft mode on." add empty final attribute
        % final
    }

\title{Welchen Einfluss hat technische Affinität auf das Wohlbefinden und effektive Zusammenarbeit in remote Teams.}
\shorttitle{Einfluss von technischer Affinität}

\author{Antonio Amaddio}
\affiliation{Freie Universität Berlin \\ A\&amp;O Vertiefung, Winter 2022/23, Supervisor: Dr. Lisa Handke}

\leftheader{Amaddio}

% \abstract{\mbox{}}

\keywords{APA style, demonstration}

\authornote{
   \addORCIDlink{Daniel A. Weiss}{0000-0000-0000-0000}

  Correspondence concerning this article should be addressed to Daniel A. Weiss, Department of Educational Psychology, Counseling and
  Special Education, A University Somewhere, 123 Main St., Oneonta, NY
  13820.  E-mail: daniel.weiss.led@gmail.com}

\begin{document}
\maketitle
Einleitung (1/2 Seite)
• Thematischer Einstieg, aus der Relevanz des Themas klar wird, d.h. worum geht es und
warum ist es wichtig
z.B. Rasante Zunahme von Home Office durch die Corona-Pandemie; dabei profitieren Beschäftigte nicht nur von der Flexibilität, sondern werden durch die Entgrenzung von Privat- und Berufsleben vor neue Herausforderungen gestellt; Folgen können Rollenkonflikte, mangelndes Detachement/Erholung und somit psychische und physische Belastung sein

\section{Theoretischer Hintergrund + Herleitung der Fragestellung (1 – 1 1⁄2 Seiten)}

Relevante Konstrukte noch näher erläutern; was wissen wir schon hierüber
• Hier sollten auch für die spätere Exploration/Erklärung des Falles relevante
Modelle/Theorien/Studienergebnisse eingeführt werden
• Darstellung Fragestellung (ggf. mit übergeordneter/Gruppenfragestellung mit Überleitung zu
individueller Fragestellung)
z.B. Zahlreiche (meta-)analytische Befunde zeigen, dass Telearbeit/Home Office durch räumliche und oftmals auch zeitliche Flexibilität die wahrgenommene Autonomie erhöht; Autonomie ist ein wichtiges Charakteristikum gängiger arbeitspsychologischer Theorien und trägt laut dieser zu erhöhter Motivation und Wohlbefinden bei; gleichzeitig gibt es Indikationen dafür, dass Autonomie nicht immer nur förderlich ist und im Überfluss zu Überforderung führt und somit ab einem gewissen Ausmaß Wohlbefinden und Motivation auch reduzieren kann.
Auch wenn die Arbeit im Home Office im Idealfall die Vereinbarkeit zwischen Berufs- und Privatleben fördert [indem...], geraten im Home Office arbeitende Personen oftmals in einen Rollenkonflikt zwischen beruflichen und privaten Anforderungen (insb. Family-to-work conflict). Anhand des im Folgenden dargestellten Falles soll dementsprechend erörtert werden (1) welche Faktoren trotz oder womöglich auch der erhöhten Autonomie zu einer Überforderung hinsichtlich der Vereinbarung von beruflichen und privaten Anforderungen führen und (2) wie sich dies auf das Wohlbefinden auswirken kann [= übergeordnete Fragestellungen]. Fokussiert werden soll sich dabei auf ... [individuelle Fragestellungen]

\section{Vorstellung des Falles (1⁄2 - 1 Seite)}
 Anonymisierte Beschreibung der interviewten Person
• Wichtig ist hier neben demografischen Informationen alles, was zum Verständnis der Problematik und Fragestellung relevant ist
z.B. in welchem Bereich arbeitet die Person, wie viel Autonomie hat sie allein schon durch die Tätigkeit selbst, wie autonom ist sie überhaupt im Home Office/wie manifestiert sich ihre Autonomie im Home Office; wie sieht die private/familiäre Situation aus, welche Verpflichtungen hat die Person hier
Achtung: Selektion der interviewten Person sollte mit der Fragestellung abgestimmt sein, z.B. würde es in unserem Beispiel ggf. keinen Sinn machen, über die Folgen hoher Autonomie zu sprechen, wenn es eine Person wäre, die zwar im Home Office arbeitet, aber trotzdem wenig Autonomie besitzt, weil sie festgelegte Arbeitszeiten hat und in ihrer Arbeit nur von anderen Personen abhängig ist [dies wäre dann der Ausgangspunkt für eine ganz andere Fragestellung, z.B. wie sich Home Office mit niedrig autonomen Jobs verträgt].

\subsection{Ergebnisse/Analyse (2 - 3 Seiten)}
• Darstellung der für die Beantwortung der Fragestellung relevanten Ergebnisse
• Ergebnisse sollten nicht nur aufgezählt, sondern in irgendeiner Form
konsolidiert/kategorisiert/abstrahiert werden, durch entsprechende Strukturierung im Text und/oder in Form von Tabellen oder Abbildungen
o Dies können auch Abwandlungen bereits etablierter Modelle/Abbildungen sein
• Einordnung der Ergebnisse in gängige Modelle/Theorien bzw. Erklärung der Befunde durch
bestehende Forschung
z.B. wie manifestiert sich hohe Autonomie bei der Person, was löst dies für Herausforderungen aus, die sich auf diese erhöhte Autonomie auswirken (z.B. die Erwartungshaltung, durch die hohe Flexibilität könnten „zwischendurch“ auch leicht Einkäufe erledigt werden; ständige Unterbrechungen durch Familienmitglieder, die dazu führen, dass man in seinem Arbeitsfluss immer wieder unterbrochen wird bzw. Probleme hat, sich zu konzentrieren); wie passen diese Schilderungen zu bekannten Konstrukten aus der Forschung (z.B. Family-to-Work Conflict); welche Auswirkungen haben diese Herausforderungen/wie wirken Sie sich laut Wahrnehmung der interviewten Person selbst bzw. aus Ihrer Interpretation deren Schilderungen auf deren Wohlbefinden (und welche Facetten davon genau?) aus? Lassen sich noch weitere z.B. moderierende Faktoren aus den Schilderungen der interviewten Person identifizieren (z.B., dass die hohe Flexibilität eigentlich nur dann problematisch ist, wenn die Familienmitglieder zu Hause sind)
ACHTUNG: Ihr werdet sicherlich nicht innerhalb eures individuellen Fallberichts so viele Punkte (vgl. Abbildung) adressieren, d.h. dies sind nur Beispiele, es wäre auch völlig in Ordnung (bzw. sogar eher erwünscht), wenn ihr euch nur auf 2-3 unterschiedliche Konstrukte konzentriert und diese aber dafür mehr ausbaut.

\subsection{Fazit/Konklusion (1⁄2 Seite)}
• Beantwortung der Fragestellung
• Kurze, zusammenfassende Einordnung der Ergebnisse ins übergeordnete Thema

\printbibliography

\appendix

\end{document}

%% 
%% Copyright (C) 2019 by Daniel A. Weiss <daniel.weiss.led at gmail.com>
%% 
%% This work may be distributed and/or modified under the
%% conditions of the LaTeX Project Public License (LPPL), either
%% version 1.3c of this license or (at your option) any later
%% version.  The latest version of this license is in the file:
%% 
%% http://www.latex-project.org/lppl.txt
%% 
%% Users may freely modify these files without permission, as long as the
%% copyright line and this statement are maintained intact.
%% 
%% This work is not endorsed by, affiliated with, or probably even known
%% by, the American Psychological Association.
%% 
%% This work is "maintained" (as per LPPL maintenance status) by
%% Daniel A. Weiss.
%% 
%% This work consists of the file  apa7.dtx
%% and the derived files           apa7.ins,
%%                                 apa7.cls,
%%                                 apa7.pdf,
%%                                 README,
%%                                 APA7american.txt,
%%                                 APA7british.txt,
%%                                 APA7dutch.txt,
%%                                 APA7english.txt,
%%                                 APA7german.txt,
%%                                 APA7ngerman.txt,
%%                                 APA7greek.txt,
%%                                 APA7czech.txt,
%%                                 APA7turkish.txt,
%%                                 APA7endfloat.cfg,
%%                                 Figure1.pdf,
%%                                 shortsample.tex,
%%                                 longsample.tex, and
%%                                 bibliography.bib.
%% 
%%
%% End of file `./samples/longsample.tex'.
