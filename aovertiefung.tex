%%
%% This is file `./samples/longsample.tex',
%% generated with the docstrip utility.
%%
%% The original source files were:
%%
%% apa7.dtx  (with options: `longsample')
%% ----------------------------------------------------------------------
%% 
%% apa7 - A LaTeX class for formatting documents in compliance with the
%% American Psychological Association's Publication Manual, 7th edition
%% 
%% Copyright (C) 2019 by Daniel A. Weiss <daniel.weiss.led at gmail.com>
%% 
%% This work may be distributed and/or modified under the
%% conditions of the LaTeX Project Public License (LPPL), either
%% version 1.3c of this license or (at your option) any later
%% version.  The latest version of this license is in the file:
%% 
%% http://www.latex-project.org/lppl.txt
%% 
%% Users may freely modify these files without permission, as long as the
%% copyright line and this statement are maintained intact.
%% 
%% This work is not endorsed by, affiliated with, or probably even known
%% by, the American Psychological Association.
%% 
%% ----------------------------------------------------------------------
%% 
\documentclass[man]{apa7}

\usepackage{lipsum}
\usepackage{lscape}
% https://tex.stackexchange.com/questions/2441/how-to-add-a-forced-line-break-inside-a-table-cell
\usepackage{makecell}

\usepackage[american]{babel}

\usepackage{csquotes}
\usepackage[style=apa,sortcites=true,sorting=nyt,backend=biber]{biblatex}
\DeclareLanguageMapping{american}{american-apa}
\addbibresource{bibliography.bib}

% ellipses (three dots) in quatations definitions from: https://walden-family.com/public/texland/ellipses.pdf
    %dots for main text
    \def\bigdotsspace{3pt}
    %three dots
    \def\mydots{\hbox{\hspace{\bigdotsspace}.\hspace{\bigdotsspace}.\hspace{\bigdotsspace}.\hspace{\bigdotsspace}\hspace{\bigdotsspace}}}
    %I like the same size space on each side of the ellipsis as
    % is between the dots of the ellipsis
    
    \hypersetup{
        nolinks=true,
        colorlinks=false,
        linkcolor=blue,
        filecolor=magenta,      
        urlcolor=cyan,
        pdftitle={Videoconferencing and its Effect on Media Richness Perception, Well-Being and Communication Effectiveness Over Time},
        pdfpagemode=FullScreen,
        % to turn off warning: "Package hyperref Warning: Draft mode on." add empty final attribute
        % final
    }

\title{Videoconferencing and its Effect on Media Richness Perception, Well-Being and Communication Effectiveness Over Time}
% title 2 \title{Changes in Media Richness Perception, Well-Being and Communication Effectiveness in Video-based Communication Over Time}
\shorttitle{Videoconferencing}

\author{Antonio Amaddio}
\affiliation{Freie Universität Berlin \\ A\&O Vertiefung, Winter 2022/23, Supervisor: Dr. Lisa Handke}

\leftheader{Amaddio}

% \abstract{\mbox{}}

\keywords{Media Richness, Compensatory Adaptation, Media Naturalness, Channel Expansion, Well-Being, Communication Effectiveness}

%\authornote{
%   \addORCIDlink{Antonio Amaddio}{0000-0000-0000-0000}

  %Correspondence concerning this article should be addressed to Daniel A. Weiss, Department of Educational Psychology, Counseling and
  %Special Education, A University Somewhere, 123 Main St., Oneonta, NY
  %13820.  E-mail: daniel.weiss.led@gmail.com}

\begin{document}
\maketitle
Pandemic related changes in work-life have sparkled a flood of research investigating the psychological effects of remote work. Videoconferencing turns out to be one of the most used communication media to replace a risky face-to-face meeting with a non-risky video based alternative \parencite{Riedl2021}.  Research shares the belief that videoconferencing has a fatiguing psychological effect on its users. The so called \textit{Zoom-Fatigue} seems to be a common reaction to the new media use. The daunting effect is attributed to cue characteristics which are not natural compared to a face-to-face conversation \parencite{Riedl2021}. Our perception has been evolutionary evolved by natural social communication and its cues (verbal, mimic, gesture etc.). If humans are capable to evolve in that sense, why wouldn't humans also adjust to new communication characteristics over time? Therefore, I investigate the experience of one individual in a single case study to explore potential changes in perception and behavior towards videoconferencing over time.

\section{Theoretical Background}

In 1983, \citeauthor{daft1983information} proposed a theory that describes a media’s ability to support communication on a richness scale. That is, how helpful the use of e.g. a telephone is in conveying information required to "reduce uncertainty and clarify ambiguity" (p 5). Media is rated from rich to lean, where Face-to-Face is the richest and numerical formal computer output is lowest in richness. The theory objectively assigns richness to a media. Looking at today’s fully/hybrid remote organizations, where no or only some face-to-face communication is possible, \citeauthor{daft1983information}'s richness categories seem questionable. Given that, teams in these organizations are (1) still functioning and (2) provide enough work for the organization to survive economically.

According to \citeauthor{Kock2005}, this lack of diminished productivity is explained by the workers' effort to compensate for the hindering characteristics of the medium. \citeauthor{Kock2005} suggests that the media barriers increase communication ambiguity and require additional cognitive effort to process relevant pieces of information. He calls workers' behavior to "overcompensate" with lean media information processing: Compensatory Adaptation (p. 270). This case study explores factors of ambiguity and factors of increased cognitive effort in video-based communication. It will be operationalized by interviewing the well-being and communication effectiveness of one subject.

The remote workers compensatory adaptation effort clearly shapes how they perceive video conferencing's richness initially. It would make sense to assume that remote workers get used to the new way to communicate over time. This means that they might update their appraised richness of video conferencing, by positive/negative experiences with it. Controversy to \citeauthor{daft1983information} the assumptions of \cite{Carlson1999} are adopted.

The authors argue that richness is a personal (subjective) evaluation that one derives from experience. More specifically, experience with the (1) communication partner, (2) channel (media), (3) its topic and the (4) organizational context form a knowledge base. This knowledge can be used to adequately decode information even though the communication media does not naturally support to transfer relevant cues. Similarly, data can be enriched from the sender by encoding information that alleviates understanding by the receiver–which could not be inferred rationally from the raw media message. To summarize: Experience leads to knowledge and shapes the perceived media richness. The question is: how does it do it? The following case study explores which de-/encoding experience has what effect on perceived media richness.

Most studies, finding "Zoom fatigue", have been nomothetic, one-time data collections. This case study confronts generalized characteristics of this effect and explores the underlying changes qualitatively, over time by examining (1) perceived media richness, (2) well-being and (3) communication effectiveness.

\section{Case Study}

The interviewed person is of female sex, 34 years old, married and has one son (17 months). She graduated from a hotel management school, and worked in hotels as a senior convention sales manager. Because of the pandemic crisis and flat hotel booking rates, she moved to a more secure job in a technology start-up 2.5 years ago, where she has been on parental leave for 12 months. The company uses \textit{Gather.town} as its primary video-conferencing tool. She spends 25\% of her time at work communicating in Gather.

The interview took place on 01-30-2023 at 10:30am and lasted $\approx$55 minutes. The interviewee reported preferring face-to-face conversations and to not spend more time than necessary in front of a computer or smartphone. Digital communication in her previous work experience was limited to email and phone. She was selected because she represents an en/decoding expert in natural face-to-face conversations, but a person with a novice-level of videoconferencing skills in a remote work scenario. To relate to the subject, the pseudonym Jane D. will be used.

\subsection{Results}

The baseline for perceived videoconferencing richness was high at the beginning. Jane assumed that videoconferencing shares the same communication characteristics as face-to-face meetings. In addition, she anticipated, that personal exchange would be less overshadowed with commuting distress.

The reported results have been assembled into three tables, separated by each outcome: (1) perceived media richness, (2) well-being, (3) communication effectiveness. The problems of communicating with the Gather videoconferencing software were categorized by its ambiguity (first column). The boxes to the right show related negative experiences (second column), how it was intended to be improved (fourth column) and how it affected the outcome over time (sixth and seventh columns).

The activity status in Gather shows whether a person is (a) available, (b) busy or (c) doesn't want to be disturbed. In the beginning, Jane often experienced that it was set incorrectly. She reached out to another person–by virtually walking over to the desk–and found no one. After several occurrences, she began to react angrily to the coworkers' sloppy behavior (see ambiguity A, Table 1). It also effected the teams' communication effectiveness negatively, as she had to find another way to reach out to someone if the activity status has not been updated (see Table 3). She compensated by using a chat application (Slack) to message the person to get back to her when they were actually be available again. This became the new default route. At this point, it no longer negatively impacted her well-being and communication effectiveness, as this new functional compensation strategy overcame the dysfunctional experience (see T1 vs T2 in Table 2 and 3). It did, however, reduce her perceived media richness because the activity status turned out to be unreliable (see T1 vs T2 in Table 1). 

Interestingly, Jane herself was not using the activity status correctly. In Gather one can talk directly to the target person if the other person has not set the status to "do not disturb". It occurred to her a few times that colleagues disturbed her. This verbal interruption was perceived as very obtrusive and left her nervous, anticipating the next obtrusive behavior of a teammate (see Obtrusiveness in Table 2). In most cases she was interrupted regarding a different topic that she was working on. It costed her a lot of energy to cognitively switch the topic or gave a false answer where she had to call back to clarify the mistake afterwards. That shows a negative effect on the communication effectiveness (see Table 3). She solved this problem by making it clear to everyone that she prefers to be notified first, before being visited virtually. This compensation strategy reversed the negative effect on her nervosity level and also the team's communication ineffectiveness (see Table 2 and 3). The dysfunctional experience had a lasting negative effect on her perceived media richness (see Table 1). She mentioned that in real life, it's common sense to knock on a door before entering a room or visually attract someone's attention instead of "shouting in their ear without warning". Her explanation shows that manually updating an activity status is not natural. In an office, one would see that the target person is either not present, busy or available to chat. This cue is not given naturally by Gather and is in line with the Media Naturalness Theory of \parencite{Kock2005}. The theory attributes communication ambiguity to the medias suppression of natural cues.

Another problem was that the larger the group in a videoconference, the more likely people were to accidentally interrupt each other. She said it was just not obvious who was going to speak next (See ambiguity B). This was especially true in situations where someone said something humorous that everyone wanted to respond to, verbally. In dyadic conversations, interruptions were most often caused by broadband interference. The latter represented a problem which could not be improved in most cases. This led to less enjoyable conversations with other people as one could not express itself naturally and one had to wait or repeat itself more often because of internet hick-ups. Team members coped with this negative experience by leaving longer gaps between conversational turns or simply did not respond at all. Both strategies were ineffective and left Jane's conversational satisfaction negative (see Table 2). Jane reported that at the beginning, speaking order confusion lead to communication inefficiency. However, as workers began to eliminate all non-essentials from the conversation, the effect reversed (see Table 3). Communication became more effective, trading in joy in conversations. Jane reported that videoconferencing suggests that everyone can interact any time, because everyone looks at everyone's face at the same time. In a natural setting, people would gather around a table where the distance between colleagues determines who can interact naturally. Similarly, cues who would speak next would be synchronous–i.e. unbiased by broadband. She reported that this has an acceptable but negative impact on heir perceived richness of videoconferencing (see Table 1). More recently, a new team norm in which the conversational leader shares a visual board regarding the meetings content reduced speaking-order confusion. Unfortunately, preparing such a board is time-consuming and requires more–videoconferencing caused–extra work.

This result is in line with the overcompensating finding of \citeauthor{Kock2005}'s Compensatory Adaptation. Video conferencing introduces an obstacle by this unnatural situation where the video participants try to compensate to have a conversation that is as close as possible to a face-to-face conversation. Jane's willingness to overcompensate is an important motivational constraint \parencite{Kock2001}. Jane has a general expectation that she wants to do a good job. She tackled obstacles introduces by videoconferencing to serve her expectations. Consistently, Jane's work load increased, which she perceived as rather stressful. This shows a negative impact on her well-being over time. That might be different for someone who doesn't have such high standards for themselves.

The final problem reported was ambiguity in emotional expression and appraisal (see ambiguity C in Tables 1, 2, 3). Jane struggled to take the emotional perspective of her virtual interlocutors. In face-to-face communication, she would rate herself as fairly good at it. Appraisal became even more difficult as the group-size increased. Scanning the number of faces for emotional expressions was perceived very fatiguing. She experienced that coworkers showed less mimics/gesture at first. Most likely, her co-workers were themselves busy scanning faces so their attention was not on the actual content of the conversation.

Another explanation could be that people may have been uncomfortable with the new situation, leading them to protect their thoughts with external expressions. Jane's report shows a negative effect on a her well-being as the higher cognitive load lead to distress and eventually fatigue (see Table 2). Over time, teammates accustomed to the video-setting and coped with a behavior that would be inauthentic or exaggerated in face-to-face conversations. People began to express their emotions by making heart-shaped hand gestures, pointed to their eyes to mimic crying, used digital emoticon (=emotion icon) as reactions or verbalized their affections like: "I am overly happy to work within a team like this". Jane perceived this development as rather pleasant. She likes to talk about emotions more openly. Over time this reflects a sustainable positive effect on her perceived richness towards videoconferencing. She said that it even compensated for negative effects on her stress and energy-level because she no longer has to constantly search for emotions in the old way.

The reported results for perceived media richness fit both: (1) the fixed characteristic of the original media richness theory (see \cite{daft1983information}) as well as (2) the individualistic appraisal suggested by Channel Expansion Theory (see \cite{Carlson1999}). Jane's knowledge base of functional behaviors due to her de-/encoding experiences helped her to improve problems introduced by videoconferencing. This affected her appraisal of the media's richness (variability). On the other hand, technical boundaries cannot be completely overcome (fixation). The gap between natural and virtual communication cues can be reduced by functional learning experiences, but not fully closed.

\subsection{Conclusion}

Three factors of ambiguity can be concluded from Jane's report in video communication: Activity status, Speaking order and Emotions. Over time, communication ambiguity had a mixed effect on Jane's perceived richness of videoconferencing. Ambiguity regarding the activity status (A) and speaking order (E) could be resolved, with the cost of perceiving videoconferencing as more lean. As Jane herself concluded, this could be due to the fact that this way or another, communicating via video conferencing remains more stressful. May it to transport emotions better or to handle speaking-order.

Quite interestingly, the ambiguity in emotion appraisal, eventually increased her perceived richness of videoconferencing. Despite the extra effort, the coping strategies led to a team-culture that paid more attention to verbalize/react emotionally. The initial problem led to a change in communication that is beyond fixing a technical problem–it enriched the communication quality. It can not be concluded if this is a generalizable positive change, but for Jane, to talk about emotions, to know what is going on in other people is perceived positive.

This case study shows that video conferencing richness is not adequately described by characteristics which generally are assumed to cause "Zoom fatigue". Individual experiences with videoconferencing might vary. This result suggests that either a lot of variables must be controlled or perceived media richness must be studied in a more narrowed down, specific, aspect.

The short observation of three months of Gather usage is a limitation of this case study. A longer period might show other factors and effects of communication ambiguity.

\printbibliography

\appendix
\begin{landscape}
\begin{table}
\caption{Perceived Media Richness}
\label{tab:BasicTable}
\scriptsize
\begin{tabular}{llllll} \toprule
\textbf{Communication ambiguity} & \textbf{En-/Decoding experience}                & \multicolumn{3}{l}{\textbf{Improvement Strategy}} & \textbf{Funct. Impr.}\tabfnm{a} \\ \cmidrule(r){3-5}
\multicolumn{4}{r}{\textbf{T1}} & \textbf{T2} &  \\ \midrule
Activity status (A)                & Activity status incorrect                       & Ignore status; Bypass with Slack   & -             & -           & N                               \\
                                   & Obtrusiveness / verbal interruption             & Communicate disliked behavior      & -             & -           & N                               \\
Speaking order (B)                 & Voice overlap (internet connectivity)           &                                    & -             & -           & N                               \\
                                   & Voice overlap (other reasons)                   &                                    & -             &             &                                 \\
                                   &                                                 & Pause between turn takings         &               & -           & N                               \\
                                   &                                                 & Focusation                         &               & -           & N                               \\
                                   &                                                 & Visual conversation guide          &               & -           & N                               \\
Emotions (C)                       & Difficulty in emotion interpretation &                                    & -             &             &                                 \\
                                   &                                                 & \multicolumn{2}{l}{React with Emoticon; Verbalize} & ++ (+)      & Y \\ \bottomrule                       
\end{tabular}
  \begin{tablenotes}[para,flushleft]
        {\tiny
            \tabfnt{a}Functional Improvement: The improvement strategy changes a negative impact of a en-/decoding experience into a neutral or positive effect on the outcome.
         }
    \end{tablenotes}
\end{table}

\begin{table}
\caption{Well-Being}
\label{tab:BasicTable}
\scriptsize
\begin{tabular}{lllllll} \toprule
\textbf{Communication   ambiguity} & \textbf{En-/Decoding experience}                & \textbf{Improvement Strategy}    & \multicolumn{3}{l}{\textbf{Facet}} & \textbf{Funct. Impr.}\tabfnm{a} \\ \cmidrule(r){4-6}
\multicolumn{5}{r}{\textbf{T1}} & \multicolumn{2}{l}{\textbf{T2}} \\ \midrule
Activity status (A)                & Activity status incorrect                       & Ignore status; Bypass with Slack & Anger               & -           & + (=)       & Y                               \\
                                   &                                                 &                                  & Distress            & -           & + (=)       & Y                               \\
                                   & Obtrusiveness / verbal interruption             & Communicate disliked behavior    & Nervosity           & -           & + (=)       & Y                               \\
Speaking order (E)                 & Voice overlap (internet connectivity)           &                                  & Joy                 & -           & -           & N                               \\
                                   & Voice overlap (other reasons)                   &                                  & Joy                 & -           &             &                                 \\
                                   &                                                 & Pause between turn takings       & Joy                 &             & -           & N                               \\
                                   &                                                 & Focusation                       & Joy                 &             & -           & N                               \\
                                   &                                                 & Visual conversation guide        & Joy                 &             & -           & N                               \\
                                   &                                                 &                                  & Distress            & -           &             &                                 \\
                                   &                                                 &                                  & Distress            &             & -           & N                               \\
Emotions (D)                       & Difficulty in emotion interpretation &                                  & Distress            & -           &             &                                 \\
                                   &                                                 &                                  & Fatigue             & -           &             &                                 \\
                                   &                                                 &                                  & Joy                 & -           &             &                                 \\
                                   &                                                 & React with Emoticon; Verbalize   & Distress            &             & -           & N                               \\
                                   &                                                 &                                  & Fatigue             &             & + (=)       & Y                               \\
                                   &                                                 &                                  & Joy                 &             & + (=)       & Y                              \\ \bottomrule
\end{tabular}
  \begin{tablenotes}[para,flushleft]
        {\tiny
            \tabfnt{a}Functional Improvement: The improvement strategy changes a negative impact of a en-/decoding experience into a neutral or positive effect on the outcome.
         }
    \end{tablenotes}
\end{table}

\begin{table}
\caption{Communication Effectiveness}
\label{tab:BasicTable}
\scriptsize
\begin{tabular}{lllllll} \toprule
\textbf{Communication   ambiguity} & \textbf{En-/Decoding experience}                & \textbf{Improvement Strategy}    & \multicolumn{3}{l}{\textbf{Facet}} & \textbf{Funct. Impr.}\tabfnm{a} \\ \cmidrule(r){4-6}
\multicolumn{5}{r}{\textbf{T1}} & \multicolumn{2}{l}{\textbf{T2}} \\ \midrule
Activity status (A)                & Activity status incorrect                       & Ignore status; Bypass with Slack & Non-availability                       & -           & + (=)       & Y                               \\
                                   & Obtrusiveness / verbal interruption             & Communicate disliked behavior    & Expensive contextual change & -           & + (=)       & Y                               \\
Speaking order (E)                 & Voice overlap (internet connectivity)           &                                  &                                        & -           & -           &                                 \\
                                   & Voice overlap (other reasons)                   &                                  &                                        & -           &             &                                 \\
                                   &                                                 & Pause between turn takings       &                                        &             & -           & N                               \\
                                   &                                                 & Focusation                       &                                        &             & + (=)       & Y                               \\
                                   &                                                 & Visual conversation guide        &                                        &             & + (=)       & Y                               \\
                                   &                                                 &                                  & Extra effort                           &             & -           & N                               \\
                                   &                                                 &                                  &                                        &             &             &                                 \\
Emotions (D)                       & Difficulty in emotion interpretation &                                  & Incapable of emotional appraisal       & -           &             &                                 \\
                                   &                                                 &                                  &                                        & -           &             &                                 \\
                                   &                                                 &                                  &                                        & -           &             &                                 \\
                                   &                                                 & React with Emoticon; Verbalize   & Extra effort                           &             & -           & N                               \\
                                   &                                                 &                                  & Capable of emotional appraisal         &             & + (=)       & Y                               \\
                                   &                                                 &                                  & Emotional perspective taking  &             & + (=)       & Y                               \\
                                   &                                                 &                                  & Joy                                    &             & + (=)       & Y                                 \\ \bottomrule
\end{tabular}
  \begin{tablenotes}[para,flushleft]
        {\tiny
            \tabfnt{a}Functional Improvement: The improvement strategy changes a negative impact of a en-/decoding experience into a neutral or positive effect on the outcome.
         }
    \end{tablenotes}
\end{table}

\end{landscape}
\end{document}

%% 
%% Copyright (C) 2019 by Daniel A. Weiss <daniel.weiss.led at gmail.com>
%% 
%% This work may be distributed and/or modified under the
%% conditions of the LaTeX Project Public License (LPPL), either
%% version 1.3c of this license or (at your option) any later
%% version.  The latest version of this license is in the file:
%% 
%% http://www.latex-project.org/lppl.txt
%% 
%% Users may freely modify these files without permission, as long as the
%% copyright line and this statement are maintained intact.
%% 
%% This work is not endorsed by, affiliated with, or probably even known
%% by, the American Psychological Association.
%% 
%% This work is "maintained" (as per LPPL maintenance status) by
%% Daniel A. Weiss.
%% 
%% This work consists of the file  apa7.dtx
%% and the derived files           apa7.ins,
%%                                 apa7.cls,
%%                                 apa7.pdf,
%%                                 README,
%%                                 APA7american.txt,
%%                                 APA7british.txt,
%%                                 APA7dutch.txt,
%%                                 APA7english.txt,
%%                                 APA7german.txt,
%%                                 APA7ngerman.txt,
%%                                 APA7greek.txt,
%%                                 APA7czech.txt,
%%                                 APA7turkish.txt,
%%                                 APA7endfloat.cfg,
%%                                 Figure1.pdf,
%%                                 shortsample.tex,
%%                                 longsample.tex, and
%%                                 bibliography.bib.
%% 
%%
%% End of file `./samples/longsample.tex'.
